\documentclass[12pt,a4paper,twoside]{article}
\usepackage{amsmath}
\usepackage{amsthm}
\usepackage{amsfonts}
\usepackage{amssymb}
\usepackage{eucal}
\usepackage[all]{xy}

\pagestyle{headings}
\setlength{\topmargin}{-15mm}
\setlength{\headheight}{1cm}
\setlength{\textheight}{25cm}
\setlength{\topskip}{-2cm}
\setlength{\oddsidemargin}{0cm}
\setlength{\evensidemargin}{0cm}
\setlength{\textwidth}{16cm}
\newlength{\abstand}
\setlength{\abstand}{5 mm}

\def\Set{{\mathbf{Set}}}
\def\SetER{{\Set_{\text{ER}}}}
\def\SetEER{{\Set_{\text{ERR}}}}
\def\refl{{\text{refl}}}
\def\symm{{\text{symm}}}
\def\trans{{\text{trans}}}

\swapnumbers
\theoremstyle{definition}
\newtheorem{defi}{Definition}[section]
\newtheorem{bsp}[defi]{Example}
\newtheorem{bspe}[defi]{Examples}
\newtheorem{satzdefi}[defi]{Proposition/ Definition}
\newtheorem{lemma}[defi]{Lemma}
\newtheorem{lemmadefi}[defi]{Lemma/ Definition}
\newtheorem{rem}[defi]{Remark}
\newtheorem{satz}[defi]{Proposition}
\newtheorem{thm}[defi]{Theorem}
\newtheorem{cor}[defi]{Corollary}

\renewcommand{\qed}{{\bf q.e.d.}}
\renewcommand{\theenumi}{(\roman{enumi})}
\renewcommand{\labelenumi}{\theenumi}
\renewcommand{\theenumii}{(\arabic{enumii})}
\renewcommand{\labelenumii}{\theenumii}
\renewcommand{\thefootnote}{\fnsymbol{footnote}}

\hbadness=10000

%%%%%%%%%%%%%%%%%%%%%%%%%%%%%%%%%%%%%%%%%%%%%%%%%%%%%%%%%%%%%%%%%%%%%%%%%%%%%%%%%%%%%%
%%%%%%%%%%%%%%%%%%%%%%%%%%%%%%%%%%%%%%%%%%%%%%%%%%%%%%%%%%%%%%%%%%%%%%%%%%%%%%%%%%%%%%
%%%%%%%%%%%%%%%%%%%%%%%%%%%%%%%%%%%%%%%%%%%%%%%%%%%%%%%%%%%%%%%%%%%%%%%%%%%%%%%%%%%%%%
%%%%%%%%%%%%%%%%%%%%%%%%%%%%%%%%%%%%%%%%%%%%%%%%%%%%%%%%%%%%%%%%%%%%%%%%%%%%%%%%%%%%%%
%%%%%%%%%%%%%%%%%%%%%%%%%%%%%%%%%%%%%%%%%%%%%%%%%%%%%%%%%%%%%%%%%%%%%%%%%%%%%%%%%%%%%%
%%%%%%%%%%%%%%%%%%%%%%%%%%%%%%%%%%%%%%%%%%%%%%%%%%%%%%%%%%%%%%%%%%%%%%%%%%%%%%%%%%%%%%
%%%%%%%%%%%%%%%%%%%%%%%%%%%%%%%%%%%%%%%%%%%%%%%%%%%%%%%%%%%%%%%%%%%%%%%%%%%%%%%%%%%%%%
%%%%%%%%%%%%%%%%%%%%%%%%%%%%%%%%%%%%%%%%%%%%%%%%%%%%%%%%%%%%%%%%%%%%%%%%%%%%%%%%%%%%%%
%%%%%%%%%%%%%%%%%%%%%%%%%%%%%%%%%%%%%%%%%%%%%%%%%%%%%%%%%%%%%%%%%%%%%%%%%%%%%%%%%%%%%%
%%%%%%%%%%%%%%%%%%%%%%%%%%%%%%%%%%%%%%%%%%%%%%%%%%%%%%%%%%%%%%%%%%%%%%%%%%%%%%%%%%%%%%
%%%%%%%%%%%%%%%%%%%%%%%%%%%%%%%%%%%%%%%%%%%%%%%%%%%%%%%%%%%%%%%%%%%%%%%%%%%%%%%%%%%%%%
%%%%%%%%%%%%%%%%%%%%%%%%%%%%%%%%%%%%%%%%%%%%%%%%%%%%%%%%%%%%%%%%%%%%%%%%%%%%%%%%%%%%%%

\begin{document}

\pagestyle{headings}

\title{Protop - Topos Programming}
\author{Lars Br\"unjes}
\date{\today}
\maketitle

%%%%%%%%%%%%%%%%%%%%%%%%%%%%%%%%%%%%%%%%%%%%%%%%%%%%%%%%%%%%%%%%%%%%%%%%%%%%%%%%%%%%%%
%%%%%%%%%%%%%%%%%%%%%%%%%%%%%%%%%%%%%%%%%%%%%%%%%%%%%%%%%%%%%%%%%%%%%%%%%%%%%%%%%%%%%%
%%%%%%%%%%%%%%%%%%%%%%%%%%%%%%%%%%%%%%%%%%%%%%%%%%%%%%%%%%%%%%%%%%%%%%%%%%%%%%%%%%%%%%
%%%%%%%%%%%%%%%%%%%%%%%%%%%%%%%%%%%%%%%%%%%%%%%%%%%%%%%%%%%%%%%%%%%%%%%%%%%%%%%%%%%%%%
%%%%%%%%%%%%%%%%%%%%%%%%%%%%%%%%%%%%%%%%%%%%%%%%%%%%%%%%%%%%%%%%%%%%%%%%%%%%%%%%%%%%%%
%%%%%%%%%%%%%%%%%%%%%%%%%%%%%%%%%%%%%%%%%%%%%%%%%%%%%%%%%%%%%%%%%%%%%%%%%%%%%%%%%%%%%%
%%%%%%%%%%%%%%%%%%%%%%%%%%%%%%%%%%%%%%%%%%%%%%%%%%%%%%%%%%%%%%%%%%%%%%%%%%%%%%%%%%%%%%
%%%%%%%%%%%%%%%%%%%%%%%%%%%%%%%%%%%%%%%%%%%%%%%%%%%%%%%%%%%%%%%%%%%%%%%%%%%%%%%%%%%%%%
%%%%%%%%%%%%%%%%%%%%%%%%%%%%%%%%%%%%%%%%%%%%%%%%%%%%%%%%%%%%%%%%%%%%%%%%%%%%%%%%%%%%%%
%%%%%%%%%%%%%%%%%%%%%%%%%%%%%%%%%%%%%%%%%%%%%%%%%%%%%%%%%%%%%%%%%%%%%%%%%%%%%%%%%%%%%%
%%%%%%%%%%%%%%%%%%%%%%%%%%%%%%%%%%%%%%%%%%%%%%%%%%%%%%%%%%%%%%%%%%%%%%%%%%%%%%%%%%%%%%
%%%%%%%%%%%%%%%%%%%%%%%%%%%%%%%%%%%%%%%%%%%%%%%%%%%%%%%%%%%%%%%%%%%%%%%%%%%%%%%%%%%%%%

\section{Explicit equivalence relations}

\vspace{\abstand}

As a warm-up, let us look at \emph{sets with equivalence relations}:

\begin{defi}\label{defSetER}
    Let us consider the category $\SetER$ of \emph{sets with equivalence relations}:
    \emph{Objects} are pairs $(A,\sim_A)$, consisting of a set $A$ and an
    equivalence relation $\sim_A$ on $A$,
    and a morphism $\left[f\right]:\ (A,\sim_A)\rightarrow(B,\sim_B)$ is an
    equivalence class of functions
    $f:\ A\rightarrow B$ that respect the equivalence relations, i.e.
    \[
        \forall x,y\in A:\ x\sim_Ay\Rightarrow fx\sim_Bfy,
    \]
    where two such functions $f$ and $g$ are \emph{equivalent} iff
    \[
        \forall x\in A:\ fx\sim_Bgx.
    \]
\end{defi}

\vspace{\abstand}

\begin{lemma}\label{lemmaSetERSet}
    Let $\Set\rightarrow\SetER$ be the functor given by sending a set $A$ to
    the pair $(A,=_A)$ (with equivalence given by equality). This is an
    equivalence of categories, and its inverse is given by sending a pair
    $(A,\sim_A)$ to $A/\sim_A$, the set of equivalence classes.
\end{lemma}

\vspace{\abstand}

\begin{proof}
    This is trivial and immediately follows from the definitions of $\SetER$
    and of equivalence relations and equivalence classes.
\end{proof}

\vspace{\abstand}

An equivalence relation on a set $A$ tells us which elements of that set are
equivalent, but it carries no further information. In particular, it does not
tell us \emph{why} two elements of $A$ are equivalent.

We will later need such extra information, which leads us to the idea of
\emph{enriching} the equivalence relations and to the study of the category
$\SetEER$ of sets with \emph{enriched} (or \emph {explicit}) equivalence
relations:

\begin{defi}\label{defEERSetEER}
    We will first define what we mean by an "enriched equivalence relation"
    and the use those enriched equivalence relations to define the category
    $\SetEER$ in analogy to $\SetER$ above.
    \begin{enumerate}
        \item\label{defEER}
            Let $A$ be a set. An \emph{enriched equivalence relation (EER)} on $A$
            is given by the following data:
            \begin{enumerate}
                \item
                    A set $R_A$ and two maps $l_A,r_A:P_A\rightarrow A$.
                    (The interpretation of an element $p$ of $P_A$ is that of a
                    "proof" that $l_Ap$ and $r_Ap$ are equivalent.)
                \item
                    A map $\refl_A:A\rightarrow P_A$ that is a section of
                    both $l_A$ and $r_A$, i.e.
                    \[
                        l_A\,\refl_A\ =r_A\,\refl_A\ =1_A.
                    \]
                    (This map represents \emph{reflexivity}, the fact that
                    each element of $A$ is equivalent to itself.)
                \item
                    A map $\symm_A:P_A\rightarrow P_A$ satisfying
                    \[
                        l_A\,\symm_A\ =\ r_A
                        \;\;\;\text{and}\;\;\;
                        r_A\,\symm_A\ =\ l_A.
                    \]
                    (This is \emph{symmetry}, stating that for $x,y\in A$,
                    $x$ is equivalent to $y$ iff $y$ is equivalent to $x$.)
                \item
                    A \emph{partial} map $\trans_A:P_A\times P_A\rightarrow P_A$,
                    defined for pairs $(p,q)$ with $r_Ap=l_Aq$, such that
                    \[
                        \forall p,q\in P_A:\ r_Ap=l_Aq\
                        \Rightarrow\
                        l_A\,\trans_A(p,q)\ = l_Ap
                        \;\;\wedge\;\;
                        r_A\,\trans_A(p,q)\ = r_Aq.
                    \]
                    (This is \emph{transitivity}, stating that for $x,y,z\in A$,
                    if $x$ and $y$ are equivalent as well as $y$ and $z$, then
                    so are $x$ and $z$.)
            \end{enumerate}
        \item\label{defSetEER}
            We now define the category $\SetEER$ as follows: \emph{Objects}
            of $\SetEER$ are pairs $(A,P_A)$, where $A$ is a set
            and $P_A=(P_A,l_A,r_A,\refl_A,\symm_A,\trans_A)$ 
            is an enriched equivalence relation on $A$.
            \emph{Morphisms} $\left[(f,f_P)\right]:\,(A,P_A)\rightarrow(B,P_B)$
            are equivalence classes of pairs $(f,f_P)$, where
            $f:A\rightarrow B$ and $f_P:A_P\rightarrow B_P$ are maps that
            "respect" the enriched equivalence relations in the following sense:
            \[
                l_Bf_P\ =\ fl_A
                \;\;\;\text{and}\;\;\;
                r_Bf_P\ =\ fr_A.
            \]
            Two pairs $(f,f_P)$ and $(g,g_P)$ are \emph{equivalent} iff
            there is a map $p_{fg}:A\rightarrow P_B$ with
            \[
                l_b\,p_{fg}\ =\ f
                \;\;\;\text{and}\;\;\;
                r_b\,p_{fg}\ =\ g.
            \]
            (The interpretation of morphisms is that of maps which
            respect equivalence. Two maps define the same morphism if they map
            each element to equivalent elements.)
    \end{enumerate}
\end{defi}

\vspace{\abstand}

\begin{rem}\label{remfsuperfluous}
    For a morphism $[(f,f_P)]:(A,P_A)\rightarrow(B,P_B)$, the map $f$
    is uniquely determined by $f_P$:
    \[
        f\ =\ 
        l_B\,f_P\,\refl_A\ =\ 
        r_B\,f_P\,\refl_A,
    \]
    so we could have defined morphisms in $\SetEER$ solely in terms of $f_P$.
    But $f$ is the "underlying", induced map on sets, so we prefer to mention
    it explicitly.
\end{rem}

\vspace{\abstand}

%%%%%%%%%%%%%%%%%%%%%%%%%%%%%%%%%%%%%%%%%%%%%%%%%%%%%%%%%%%%%%%%%%%%%%%%%%%%%%%%%%%%%%
%%%%%%%%%%%%%%%%%%%%%%%%%%%%%%%%%%%%%%%%%%%%%%%%%%%%%%%%%%%%%%%%%%%%%%%%%%%%%%%%%%%%%%
%%%%%%%%%%%%%%%%%%%%%%%%%%%%%%%%%%%%%%%%%%%%%%%%%%%%%%%%%%%%%%%%%%%%%%%%%%%%%%%%%%%%%%
%%%%%%%%%%%%%%%%%%%%%%%%%%%%%%%%%%%%%%%%%%%%%%%%%%%%%%%%%%%%%%%%%%%%%%%%%%%%%%%%%%%%%%
%%%%%%%%%%%%%%%%%%%%%%%%%%%%%%%%%%%%%%%%%%%%%%%%%%%%%%%%%%%%%%%%%%%%%%%%%%%%%%%%%%%%%%
%%%%%%%%%%%%%%%%%%%%%%%%%%%%%%%%%%%%%%%%%%%%%%%%%%%%%%%%%%%%%%%%%%%%%%%%%%%%%%%%%%%%%%
%%%%%%%%%%%%%%%%%%%%%%%%%%%%%%%%%%%%%%%%%%%%%%%%%%%%%%%%%%%%%%%%%%%%%%%%%%%%%%%%%%%%%%
%%%%%%%%%%%%%%%%%%%%%%%%%%%%%%%%%%%%%%%%%%%%%%%%%%%%%%%%%%%%%%%%%%%%%%%%%%%%%%%%%%%%%%
%%%%%%%%%%%%%%%%%%%%%%%%%%%%%%%%%%%%%%%%%%%%%%%%%%%%%%%%%%%%%%%%%%%%%%%%%%%%%%%%%%%%%%
%%%%%%%%%%%%%%%%%%%%%%%%%%%%%%%%%%%%%%%%%%%%%%%%%%%%%%%%%%%%%%%%%%%%%%%%%%%%%%%%%%%%%%
%%%%%%%%%%%%%%%%%%%%%%%%%%%%%%%%%%%%%%%%%%%%%%%%%%%%%%%%%%%%%%%%%%%%%%%%%%%%%%%%%%%%%%
%%%%%%%%%%%%%%%%%%%%%%%%%%%%%%%%%%%%%%%%%%%%%%%%%%%%%%%%%%%%%%%%%%%%%%%%%%%%%%%%%%%%%%

\bibliographystyle{alpha}
\bibliography{Literatur}

\end{document}
